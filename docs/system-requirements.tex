\documentclass[12pt, a4paper, dutch]{article}
\usepackage[margin=1in]{geometry}

\usepackage{float}
\usepackage{babel}
\usepackage{siunitx}
\usepackage{amsmath}
\usepackage{csquotes}
\usepackage{parskip}
\usepackage{unicode-math}
\usepackage{fontspec}
\usepackage{tabularx}
\usepackage{booktabs}
\usepackage{needspace}
\usepackage{hyperref}
% \usepackage[backend=biber,
% 	bibencoding=utf8,
% 	style=apa
% ]{biblatex}

\setmainfont{TeX Gyre Schola}
\setmathfont{TeX Gyre Schola Math}
\sisetup{
	group-separator = {.},
	output-decimal-marker = {,}
}

\bigskipamount=7mm
\medskipamount=4mm
\parindent=0mm

\newcommand{\req}[1]{$^{\text{\ref{req:#1}}}$}

\begin{document}
System requirements \hfill \textbf{Loek Le Blansch (2180996)}\\
Project Stylofoon \hfill \today
\medskip

\section{Inleiding}

De stylofoon is een klein instrument die met behulp van een stylus bespeeld wordt. Op
de stylofoon zitten 20 contactpunten in de vorm van pianotoetsen. Het uiteinde van de
stylus bevat een elektrisch contact, die gebruikt wordt om de pianotoetsvormige
contacten aan te raken. De stylofoon heeft twee ingebouwde synthesizers, die mengbaar
zijn met behulp van een draaiknop om unieke geluiden te cre\"eren.

\section{Projectopdracht}

``Ik wil graag één muziekinstrument\req{20keys} die op twee verschillende manieren
geluid kan produceren\req{engines}, met een Stylofoon en Synthesizer. De Stylofoon
moet 20 tonen kunnen genereren\req{20keys}. De Synthesizer moet gebaseerd zijn op een
Arduino\req{engines} en 8 variabele tonen maken\req{wavforms}. De tonen die de
Synthesizer maakt moeten met draaiknoppen instelbaar zijn op frequentie\req{tuneable}
en lengte\req{sustain}. Er moet een schakelaar aanwezig zijn om te wisselen tussen
het afspelen van de Stylofoon en Synthesizer\req{mixfader}. Het geluidsignaal moet
ook visueel gemaakt worden op een VU-meter\req{vumeter}. Het volume moet ingesteld
kunnen worden met een draaiknop\req{volknob}. De status van het gehele instrument
moet visueel weergegeven worden\req{onled}. Het zou mooi zijn als in plaats van een
schakelaar er een manier is om de geluidssignalen van de Stylofoon en Synthesizer te
combineren\req{mixfader}.

De Stylofoon heeft minimaal 20 contactpunten\req{20keys} en is gebaseerd op een 555
timer\req{555}. De Synthesizer is gebaseerd op een Arduino Uno\req{engines} en heeft
minimaal 8 druktoetsen\req{wavswitch} en 2 draaiknoppen\req{sustain}\req{tuneable}.
Verder is er een draaiknop om het volume te regelen\req{volknob}, een
luidspreker-aansluiting\req{lineout} en een VU-meter\req{vumeter}. Alleen pinnen
A0-A5 van de Arduino Uno mogen gebruikt worden\req{a05}.''

\section{Technische eisen}

In deze vereisten word er met het woord `synthesizer' alleen het elektrische
onderdeel dat een geluidssignaal produceert bedoeld, niet de stylofoon in zijn
geheel.

\begin{enumerate}
\subsection{Functionele specificaties}
	\item \label{req:20keys} \'E\'en fysiek klaviertoetsenbord bestaand uit 20
		elektrisch geleidende contacten
	\item \label{req:mixfader} Een draaiknop of fader die het geluid van de analoge en
		de digitale synthesizers mixt voordat het signaal versterkt wordt
	\item \label{req:vumeter} Een VU-meter die visueel de amplitude van het
		uigangssignaal laat zien
	\item \label{req:volknob} Een volumeknop om het uigangsvolume aan te passen
	\item \label{req:onled} Een led lampje die toont of de stylofoon aan staat
\subsection{Operationele functies}
	\item \label{req:tuneable} De individuele tonen van de analoge synthesizer zijn
		stembaar met behulp van instelpotmeters
	\item \label{req:engines} Twee synthesizers, een analoge en een digitale die met
		behulp van een Arduino Uno gemaakt wordt
	\begin{enumerate}
		\item \label{req:wavforms} De digitale synthesizer produceert acht verschillende
			golfvormen
		\item \label{req:wavswitch} De digitale synthesizer heeft acht drukknoppen om
			tussen de golfvormen te wisselen
		\item \label{req:sustain} De digitale synthesizer heeft een draaiknop om de
			lengte (sustain) van de tonen aan te kunnen passen
	\end{enumerate}
	\item \label{req:lineout} De audio uitgang van de stylofoon is op line-level, niet
		op headphone-level
\subsection{Ontwerpbeperkingen}
	\item \label{req:555} De analoge synthesizer is gebaseerd op een 555 timer
	\item \label{req:a05} Alleen pinnen A0..A5 van de Arduino worden gebruikt
	\item \label{req:lm3914} Er wordt een LM3914 gebruikt voor de VU-meter
\subsection{Randvoorwaarden}
	\item Sommige componenten worden voorgeschreven
	\item Diverse componenten zoals de gaatjesprint, druktoetsen en leds worden door de
		labbeheerders geleverd
	\item Componenten die niet in voorraad zijn worden in overleg met de labbeheerders
		gekozen
	\item Er worden IC-voetjes gebruikt om de ledbar van de VU-meter, en andere IC's te
		monteren
\end{enumerate}

\section{Vragen}

\begin{itemize}
	\item Moet de uitgang van de stylofoon een preamp hebben, of word de stylofoon
		aangesloten op een externe versterker?
	\item In de projectopdracht staat ``\emph{De tonen die de Synthesizer maakt moeten
		met draaiknoppen instelbaar zijn op frequentie en lengte}''
	\begin{enumerate}
		\item Word er hier met `Synthesizer' alleen de Arduino-synthesizer of allebei de
			interne synthesizers bedoeld?
		\item Word hier bedoeld dat alle tonen tegelijkertijd hoger, of lager worden, of
			dat elke toon individueel stembaar is (met behulp van instelpotmeters)?
	\end{enumerate}
	\item Moet de VU-meter het geluidssignaal voor of na het door de volumeknop
		aangepast is weergeven?
	\item In de projectopdracht staat ``\emph{De Synthesizer (...) [moet] 8 variabele
		tonen maken}'', word hier bedoeld dat de Arduino maar geluid hoeft te maken voor
		8 van de 20 toetsen, of dat er 8 unieke golfvormen geproduceerd moeten worden?
\end{itemize}

\end{document}

