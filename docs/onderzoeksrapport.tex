\documentclass[12pt, a4paper, dutch]{article}
\usepackage[margin=1in]{geometry}

\usepackage{float}
\usepackage{babel}
\usepackage{siunitx}
\usepackage{amsmath}
\usepackage{csquotes}
\usepackage{parskip}
\usepackage{unicode-math}
\usepackage{fontspec}
\usepackage{tabularx}
\usepackage{booktabs}
\usepackage{needspace}
\usepackage{hyperref}
% \usepackage[backend=biber,
% 	bibencoding=utf8,
% 	style=apa
% ]{biblatex}

\setmainfont{TeX Gyre Schola}
\setmathfont{TeX Gyre Schola Math}
\sisetup{
	group-separator = {.},
	output-decimal-marker = {,}
}

\newcommand{\s}{$^{\sharp}$}
\newcommand{\sub}[1]{$_{\text{#1}}$}

\bigskipamount=7mm
\medskipamount=4mm
\parindent=0mm

\begin{document}
Onderzoeksrapport \hfill \textbf{Loek Le Blansch (2180996)}\\
Project Stylofoon \hfill \today
\medskip

\tableofcontents

\section{Veelgebruikte eenheden}

\textit{Welke eenheden worden er gebruikt om stromen en spanningen in audio-circuits
te meten?}

\section{555 Timer configuratie}

\textit{Hoe configureer je een 555 timer op een manier waarop de spanning of
weerstand die de frequentie regelt meetbaar is zonder de frequentie te be\"invloeden,
en de frequentie uitrekenbaar is?}

In het ontwerp gaat het keyboard van de noot f3 tot c5. Hier is een tabel met
nootnamen, absolute frequenties en de invoerspanning die de 555 oscillator nodig
heeft om die frequentie te produceren:

\begin{figure}[H]
\centering
\begin{tabular}{ccc}
\toprule
Noot & $f$ [\si{\hertz}] & V\sub{in} [\si{\volt}]\\
\midrule
f3   & \num{174.6} & \num{0.0} \\
f\s3 & \num{185.0} & \num{0.0} \\
g3   & \num{196.0} & \num{0.0} \\
g\s3 & \num{207.7} & \num{0.0} \\
a3   & \num{220.0} & \num{0.0} \\
a\s3 & \num{233.1} & \num{0.0} \\
b3   & \num{246.9} & \num{0.0} \\
c4   & \num{261.6} & \num{0.0} \\
c\s4 & \num{277.2} & \num{0.0} \\
d4   & \num{293.7} & \num{0.0} \\
d\s4 & \num{311.1} & \num{0.0} \\
e4   & \num{329.6} & \num{0.0} \\
f4   & \num{349.2} & \num{0.0} \\
f\s4 & \num{370.0} & \num{0.0} \\
g4   & \num{392.0} & \num{0.0} \\
g\s4 & \num{415.3} & \num{0.0} \\
a4   & \num{440.0} & \num{0.0} \\
a\s4 & \num{466.2} & \num{0.0} \\
b4   & \num{493.9} & \num{0.0} \\
c5   & \num{523.3} & \num{0.0} \\
\bottomrule
\end{tabular}
\caption{Noot-frequentie lookup-table}
\end{figure}

Uit deze tabel is te zien dat de frequentie tussen \SI{174.6}{\hertz} en
\SI{523.3}{\hertz} moet kunnen vari\"eren. <hier een stuk over hoe je die spanningen
berekent ofzo>. Omdat er een weerstandsladder gebruikt word voor het keyboard, moet
er een manier zijn om te compenseren voor de toleranties van de weerstanden. Hiervoor
worden instelpotmeters in serie gesoldeerd die de weerstandswaarde kunnen laten
vari\"eren buiten de tolerantie van elke weerstand.

\section{Spanningen en stromen}

\textit{Welke voedingen kunnen gebruikt worden om alle onderdelen te voorzien van
genoeg stroom?}

Het hele systeem draait op een voedingsspanning van V\sub{CC} $=$ \SI{5.0}{\volt}.
Het zou ideaal zijn als de hele stylofoon door \'e\'en USB poort gevoed kan worden.
Een standaard USB aansluiting levert echter niet meer dan \SI{500}{\milli\ampere},
dus als alle onderdelen gecombineerd meer dan deze limiet gebruiken moet er een
andere oplossing voor de voeding gekozen worden.

\subsection{Metingen}

\begin{figure}[H]
\centering
\begin{tabular}{lccl}
\toprule
Onderdeel \\
\midrule
Arduino (audio uit)                   & V\sub{PP} & \num{0} & \si{\volt}\\
										                  & I\sub{PP} & \num{0} & \si{\milli\ampere}\\
555 (audio uit)                       & V\sub{PP} & \num{0} & \si{\volt}\\
                                      & I\sub{PP} & \num{0} & \si{\milli\ampere}\\
Functiegenerator (fatsoenlijk volume) & V\sub{PP} & \num{0} & \si{\volt}\\
																			& I\sub{PP} & \num{0} & \si{\milli\ampere}\\
\bottomrule
\end{tabular}
\caption{Gemeten stromen en spanningen}
\end{figure}

\subsection{Uitvoer}

De uitvoer van de stylofoon moet volgens de opdrachtgever `een fatsoenlijk volume'
produceren. Op internet heb ik gevonden dat een gemiddelde koptelefoonaansluiting op
een smartphone ongeveer \SI{30}{\milli\watt} bij een koptelefoon met een impedantie
van \SI{32}{\ohm} levert. Het volume van een audiosignaal is lastig te achterhalen
omdat het van veel verschillende factoren af hangt zoals de spanning, vermogen,
impedantie, en de gevoeligheid van een uitvoerapparaat.

Genoeg vermogen leveren voor oordopjes zal niet heel lastig zijn, maar omdat de
gevoeligheid van speakers heel erg vari\"eert is het lastig om een exacte waarde te
geven. Daarom ga ik er vanaf nu vanuit dat \SI{30}{\milli\watt} @ \SI{32}{\ohm}
genoeg zal zijn. Hiermee kan de stroom die de stylofoon moet leveren uitgerekend
worden als volgt:

\[
I=\sqrt{\frac{P}{R}} = \sqrt{\frac{30*10^{-3}}{32}} \approx \SI{31}{\milli\ampere}
\]

\subsection{Overzicht}

\begin{figure}[H]
\centering
\begin{tabular}{lr}
\toprule
Onderdeel & I\sub{max} [\si{\milli\ampere}]\\
\midrule
Arduino & \num{50} \\
Arduino audio uit & \num{20} \\
LM3914 & \num{0} \\
VU-display & \num{0} \\
Versterker digitale synth & \num{0} \\
Versterker analoge synth & \num{0} \\
Stroomverlies weerstandladder & \num{0} \\
Audio-uitvoer & \num{31} \\
\midrule
\hfill Totaal: & \num{500} \\
\bottomrule
\end{tabular}
\caption{Overzicht stromen}
\end{figure}

\section{Versterkers}

\textit{Hoe versterk je een audiosignaal zonder ruis of vervorming te introduceren?}

(misschien een lm386 gebruiken als versterker als die er is in het techlab)

\section{VU-display}

\textit{Hoe moet je een LM3914 toepassen om een VU-display aan te sturen?}


\end{document}

