\documentclass[12pt, a4paper, dutch]{article}
\usepackage[margin=1in]{geometry}

\usepackage{float}
\usepackage{babel}
\usepackage{siunitx}
\usepackage{amsmath}
\usepackage{csquotes}
\usepackage{parskip}
\usepackage{unicode-math}
\usepackage{fontspec}
\usepackage{tabularx}
\usepackage{booktabs}
\usepackage{needspace}
\usepackage{hyperref}
\usepackage{graphicx}
% \usepackage[backend=biber,
% 	bibencoding=utf8,
% 	style=apa
% ]{biblatex}

\setmainfont{TeX Gyre Schola}
\setmathfont{TeX Gyre Schola Math}
\sisetup{
	group-separator = {.},
	output-decimal-marker = {,}
}

\newcommand{\s}{$^{\sharp}$}
\newcommand{\sub}[1]{$_{\text{#1}}$}

\bigskipamount=7mm
\medskipamount=4mm
\parindent=0mm

\begin{document}
Onderzoeksrapport \hfill \textbf{Loek Le Blansch (2180996)}\\
Project Stylofoon \hfill \today
\medskip

\tableofcontents

\section{Veelgebruikte eenheden}

% \textit{Welke eenheden worden er gebruikt om stromen en spanningen in
% audio-circuits te meten?}

\textbf{V\sub{PP}} (voltage peak to peak)
Dit is de afstand tussen de onderste en bovenste spanningsgrens van een signaal

\textbf{V\sub{RMS}} (root mean square voltage)
Dit is de gelijkspanningequivalent van een wisselspanning. Voor een AC signaal
met een evenwichtsstand om \SI{0}{\volt} is V\sub{RMS} als volgt te berekenen:
\[ \text{V}_{\text{RMS}} = \frac{\text{V}_{\text{AC}}}{\sqrt{2}} \]

\textbf{dB} (decibel)
Decibel is de eenheid van relatief geluidsvolume. De schaalverdeling van decibel is
logaritmisch, om de manier waarop mensen volume waarnemen na te bootsen. Decibel is
een relatieve eenheid, en wordt daarom vaak gecombineerd met andere eenheden zoals
dBv (decibel voltage) of dB SPL (decibel sound pressure level) om absolute waardes
aan te geven.

\textbf{dBv} (decibel volt)
dBv is V\sub{RMS} relatief tot een spanning die \SI{1}{\milli\watt} zou verspillen
door een lading met een impedantie van \SI{600}{\ohm}. 0 dBv is gelijk aan een
spanning van $\sqrt{600 * 1 * 10^{-3}} \approx \SI{0.77}{\volt}$.

\textbf{W} (watt)
Omdat voor de amplitude van een signaal de spanning, en daarmee ook de stroom
verandert, is het gebruikelijk om op de uitgang van een versterker een vermogen te
zien. Er is alleen geen `normaal' vermogen, omdat de gevoeligheid van
uitvoerapparaten ook enorm kan vari\"eren. De IEM's die ik thuis gebruik zijn erg
gevoelig, en hebben een gevoeligheid van 112 dB SPL mW$^{-1}$.

\section{555 Timer configuratie}

% \textit{Hoe configureer je een 555 timer op een manier waarop de spanning of
% weerstand die de frequentie regelt meetbaar is zonder de frequentie te
% be\"invloeden, en de frequentie uitrekenbaar is?}

In het ontwerp gaat het keyboard van de noot f3 tot c5. Hier is een tabel met
nootnamen, absolute frequenties en de invoerspanning die de 555 oscillator nodig
heeft om die frequentie te produceren:

\begin{figure}[H]
\centering
\begin{tabular}{ccc}
\toprule
Noot & $f$ [\si{\hertz}] & V\sub{in} [\si{\volt}]\\
\midrule
f3   & \num{174.6} & \num{3.986} \\
f\s3 & \num{185.0} & \num{3.929} \\
g3   & \num{196.0} & \num{3.817} \\
g\s3 & \num{207.7} & \num{3.757} \\
a3   & \num{220.0} & \num{3.678} \\
a\s3 & \num{233.1} & \num{3.578} \\
b3   & \num{246.9} & \num{3.505} \\
c4   & \num{261.6} & \num{3.402} \\
c\s4 & \num{277.2} & \num{3.291} \\
d4   & \num{293.7} & \num{3.161} \\
d\s4 & \num{311.1} & \num{3.020} \\
e4   & \num{329.6} & \num{2.883} \\
f4   & \num{349.2} & \num{2.725} \\
f\s4 & \num{370.0} & \num{2.546} \\
g4   & \num{392.0} & \num{2.366} \\
g\s4 & \num{415.3} & \num{2.174} \\
a4   & \num{440.0} & \num{1.958} \\
a\s4 & \num{466.2} & \num{1.734} \\
b4   & \num{493.9} & \num{1.511} \\
c5   & \num{523.3} & \num{1.225} \\
\bottomrule
\end{tabular}
\caption{Noot-frequentie lookup-table}
\end{figure}

Deze metingen zijn gedaan met een voedingsspanning van \SI{5.00}{\volt}, een
weerstandwaarde van \SI{218.8}{\kilo\ohm}, en een condensator van
\SI{10.4}{\nano\farad}.

Uit deze tabel is te zien dat de frequentie tussen \SI{174.6}{\hertz} en
\SI{523.3}{\hertz} moet kunnen vari\"eren. Omdat er een weerstandsladder gebruikt
word voor het keyboard, moet er een manier zijn om te compenseren voor de toleranties
van de weerstanden. Hiervoor worden instelpotmeters in serie gesoldeerd die de
weerstandswaarde kunnen laten vari\"eren buiten de tolerantie van elke weerstand.

% Metingen frequentie bereken formule:
% 
% \begin{figure}[H]
% \centering
% \begin{tabular}{ccc}
% \toprule
% $f$ [\si{\hertz}] & R [\si{\kilo\ohm}]\\
% \midrule
% \num{373} & \num{218.8} \\
% \num{820} & \num{99.13} \\
% \num{1468} & \num{54.83} \\
% \bottomrule
% \end{tabular}
% \caption{Variabele weerstand bij condensator van \SI{10.4}{\nano\farad}}
% \end{figure}
% 
% \begin{figure}[H]
% \centering
% \begin{tabular}{ccc}
% \toprule
% $f$ [\si{\hertz}] & Q [\si{\nano\farad}]\\
% \midrule
% \num{184} & \num{21.0} \\
% \num{373} & \num{10.4} \\
% \num{676} & \num{5.54} \\
% \bottomrule
% \end{tabular}
% \caption{Variabele condensator bij weerstand van \SI{218.8}{\kilo\ohm}}
% \end{figure}

\section{Spanningen en stromen}

% \textit{Welke voedingen kunnen gebruikt worden om alle onderdelen te voorzien van
% genoeg stroom?}

Het hele systeem draait op een voedingsspanning van V\sub{CC} $=$ \SI{5.0}{\volt}.
Het zou ideaal zijn als de hele stylofoon door \'e\'en USB poort gevoed kan worden.
Een standaard USB aansluiting levert echter niet meer dan \SI{500}{\milli\ampere},
dus als alle onderdelen gecombineerd meer dan deze limiet gebruiken moet er een
andere oplossing voor de voeding gekozen worden.

% \subsection{Metingen}
% 
% \begin{figure}[H]
% \centering
% \begin{tabular}{lccl}
% \toprule
% Onderdeel \\
% \midrule
% Arduino (audio uit)                   & V\sub{RMS} & \num{0} & \si{\volt}\\
% 										                  & I\sub{RMS} & \num{0} & \si{\milli\ampere}\\
% 555 (audio uit)                       & V\sub{RMS} & \num{0} & \si{\volt}\\
%                                       & I\sub{RMS} & \num{0} & \si{\milli\ampere}\\
% Functiegenerator (fatsoenlijk volume) & V\sub{RMS} & \num{0} & \si{\volt}\\
% 																			& I\sub{RMS} & \num{0} & \si{\milli\ampere}\\
% \bottomrule
% \end{tabular}
% \caption{Gemeten stromen en spanningen}
% \end{figure}

\subsection{Uitvoer}

De uitvoer van de stylofoon moet volgens de opdrachtgever `een fatsoenlijk volume'
produceren. Op internet heb ik gevonden dat een gemiddelde koptelefoonaansluiting op
een smartphone ongeveer \SI{30}{\milli\watt} bij een koptelefoon met een impedantie
van \SI{32}{\ohm} levert. Het volume van een audiosignaal is lastig te achterhalen
omdat het van veel verschillende factoren af hangt zoals de spanning, vermogen,
impedantie, en de gevoeligheid van het uitvoerapparaat.

Genoeg vermogen leveren voor oordopjes zal niet heel lastig zijn, maar omdat de
gevoeligheid van speakers heel erg vari\"eert is het lastig om een exacte waarde te
geven. Daarom ga ik er vanaf nu vanuit dat \SI{30}{\milli\watt} @ \SI{32}{\ohm}
genoeg zal zijn. Hiermee kan de stroom die de stylofoon moet leveren uitgerekend
worden als volgt:

\[
I=\sqrt{\frac{P}{R}} = \sqrt{\frac{30*10^{-3}}{32}} \approx \SI{31}{\milli\ampere}
\]

\subsection{Overzicht}

\begin{figure}[H]
\centering
\begin{tabular}{lr}
\toprule
Onderdeel & I\sub{max} [\si{\milli\ampere}]\\
\midrule
Arduino & \num{50} \\
Arduino audio uit & \num{20} \\
555 & \num{3} \\
LM3914 & \num{13} \\
VU-display & \num{200} \\
LM386 & \num{4} \\
Audio-uitvoer & \num{31} \\
\midrule
\hfill Totaal: & \num{321} \\
\bottomrule
\end{tabular}
\caption{Overzicht stromen}
\end{figure}

\section{Versterkers}

% \textit{Hoe versterk je een audiosignaal zonder ruis of vervorming te
% introduceren?}

Om het audiosignaal hoorbaar te maken kunnen niet direct speakers aangesloten worden
op de interne synthesizers, maar deze moeten eerst versterkt worden. Audiosignalen
kunnen op verschillende manieren versterkt worden, maar voor dit project heb ik
gekozen om een LM386 te gebruiken als versterker. In de datasheet van deze IC staat
een voorbeeldopstelling:

\begin{figure}[H]
\centering
\includegraphics[
	clip,
	trim=70mm 130mm 70mm 100mm,
	page=8
]{../datasheets/lm386.pdf}
\caption{Voorbeeldopstelling LM386}
\end{figure}

In deze voorbeeldopstelling komt de audio binnen via de V\sub{IN} ingang. Deze ingang
staat in serie met een potmeter zodat het volume geregeld kan worden. De LM386 heeft
een standaardgain van 20, maar deze kan verhoogd worden door pin 1 en 8 met een
weerstand en condensator te verbinden. Hierdoor zal het geluid harder worden.

\end{document}

